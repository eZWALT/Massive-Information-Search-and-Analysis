\documentclass[12pt]{article}
\usepackage{graphicx}
\usepackage{amsmath}
\usepackage{placeins}
\usepackage{float}
\usepackage[bottom]{footmisc}
\usepackage{subcaption}

\begin{document}

\begin{titlepage}
	\centering
	\includegraphics[width=0.3\textwidth]{university.png}\par\vspace{1cm}
	{\scshape\LARGE Laboratori de CAIM \par}
	\vspace{1cm}
	{\scshape\Large Pràctica 4\par}
	\vspace{1.5cm}
	{\Large\itshape Walter J.Troiani\par}
	\vfill
	Prof: Ignasi Gómez Sebastià\par
    \vspace{1cm}
    12/11/2023 - 2023/24 Q1\par
	\vspace{1cm}

\end{titlepage}

\newpage

\section{PageRank. Detalls d'implementació}

Aquest escrit està orientat a donar context al lector de l'algorisme de PageRank emprant la matriu de Google, el qual es l'algorisme implementat en el codi. En aquest cas però, l'algorisme estarà orientat a aeroports (Nodes) i les seves rutes (Arestes) que formen un graf dirigit , els quals han estat extrets gràcies a la base de dades de OpenFlights, i no pas en pàgines web.
\\ 

%la formula debe ser fucking vectorial no man? y tambien usar un más o menos igual
L'algorisme que implementarem constarà de repetir una simple formula $p(t+1) \eq G p(t)$ on G es la matriu de Google $G = \lambda * M + \frac{1 - \lambda}{n} * J$ on J es la matriu de tot uns , M es la matriu de adjacència del graf dirigit original normalitzat per files del sistema d'avions i $\lambda$ és el factor d'amortiment. Aquesta fórmula serà aplicada fins arribar al punt de convergència, que s'arribarà quan la variació de pes entre iteracions sigui menor al factor de convergència $\epsilon$, el qual considerem que una aproximació més fidel a l'algorisme que no pas un nombre fixat de iteracions. Es essencial pel correcte funcionament i convergència de l'algorisme que el graf sigui fortament connex i apèriodic. La correctesa i eficiència de l'algorisme pot ser consultada a WikiPedia. 
\\ 

Un problema al que ens enfrentarem serán els aeroports "especials", que són aquells que o bé no tenen rutes sortints o bé no tenen rutes entrants o cap de les dues. Com a parámetres del programa tindrem 3 booleans que permetran l'eliminació de qualsevol dels 3 tipus de nodes, per a poder experimentar amb els diversos resultats.
\\ 

\begin{itemize}
    \item Aeroports Aïllats (Isolated): Són aquells que tenen rutes de sortida però no pas d'entrada. No són problemàtics a no ser que $\lambda \eq 1$, aleshores s'impedeix una de les 2 condicions essencials de l'algorisme
    \item Aeroports Pou (Terminal): Aeroports que tenen rutes d'entrada però dels quals no es pot sortir. Són problemàtics ja que el pes acaba en aquests nodes (fan de pou). 
    \item Aeroports Inaccessibles (Unreachable): Aeroports els quals no tenen cap ruta d'entrada ni sortida. 
\end{itemize}

Un problema típic del PageRank són els autobucles, però com trivialment cap avió vola al mateix aeroport podem oblidar-nos d'aquest problema.

\section{Experimentació}
En aquest algorisme entren en joc un nombre divers de paràmetres i s'han de fer certes decisions (algunes arbitràries i altres obligàtories per garantir l'acabament de l'algorisme). Caldrà revisar la casuística dels nodes inaccessibles, pous o bé ambdós.
\\

\begin{itemize}
    \item $\lambda$: Factor d'amortiment de la matriu G de Google en el rang $[0, 1]$, que regula la importància del graf original en comparació a un graf complert (i fortament connex per consequència)
    \item $\epsilon$: Constant de la condició d'aturada major a 0. El bucle acaba quan el pes de tots els nodes tenen una variació entre iteracions inferior a aquest factor, es a dir: $\forall i \in \{1,2,..,n\} : \| p_{i}(t+1) - p_{i}(t) \| < \epsilon $ atura el bucle en el instant t+1. 
\end{itemize}

Al tractar-se d'un nombre tan reduit de variables, es raonable plantejar un estudi multivariable tinguent els 2 paràmetres en compte i les dues possibles variables objectiu com poden ser el nombre d'iteracions/temps d'execució o bé el PageRank (promig, diferencia entre el máxim o mínim ...), per fer un estudi del temps de convergència entre altres.
\\

També serà rellevant estudiar els rankings obtinguts de pes pageRank a cada node i evaluar si tenen sentit amb les dades del món real. Això podem aproximar-ho qualitativament comparant els resultats que obtinguem amb dades de quins aeroports són els més concurrits, o bé els que més vols tenen/reben.
\\ 

Finalment la distribució inicial dels pesos de PageRank també podria ser investigada per a tal d'establir alguna conclusió sobre l'efecte que té al resultat dels pesos.

\subsection{Estudi de $\lambda$}

\subsection{Estudi de $\epsilon$}

\subsection{Estudi multivariable}

\subsection{Estudi d'aeroports especials}

\subsection{Estudi qualitatiu}



\section{Conclusions i Reptes}
Els problemes més comuns van ser errors en l'implementació de les classes, o bé lectures dels CSV (al ser manual) incorrectes. Evidentment com sempre, he fet la pràctica completament sol i això implica una menor quantitat de recursos humans, que han sigut comepnsats amb temps i esforç.També l'experimentació va ser un treball llarg d'implementar i verificar, però els resultats han valgut la pena. Per manca de temps no vaig poder implementar l'experimentació de distribucions inicials diferents.
\\


% what about fucking citar maldito perro panzón mileurista

\end{document}

% motherfucking ejemplo de imagen doble que si funciona

\begin{figure}[h!]
    \centering
    \hspace{-1cm}
    \begin{subfigure}[b]{0.5\textwidth}
        \includegraphics[width=\textwidth]{NroundDetroit.png}

        \caption{"Detroit"}
        \label{fig:DPPM}
    \end{subfigure}
    ~ %add desired spacing between images, e. g. ~, \quad, \qquad, \hfill etc. 
      %(or a blank line to force the subfigure onto a new line)
    \begin{subfigure}[b]{0.5\textwidth}
        \includegraphics[width=\textwidth]{NroundOhio.png}
        \caption{"Ohio"}
        \label{fig:OPPM}
    \end{subfigure}
\end{figure}
\FloatBarrier